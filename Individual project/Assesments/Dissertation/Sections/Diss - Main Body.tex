%%
% Copyright (c) 2017 - 2023, Pascal Wagler;
% Copyright (c) 2014 - 2023, John MacFarlane
%
% All rights reserved.
%
% Redistribution and use in source and binary forms, with or without
% modification, are permitted provided that the following conditions
% are met:
%
% - Redistributions of source code must retain the above copyright
% notice, this list of conditions and the following disclaimer.
%
% - Redistributions in binary form must reproduce the above copyright
% notice, this list of conditions and the following disclaimer in the
% documentation and/or other materials provided with the distribution.
%
% - Neither the name of John MacFarlane nor the names of other
% contributors may be used to endorse or promote products derived
% from this software without specific prior written permission.
%
% THIS SOFTWARE IS PROVIDED BY THE COPYRIGHT HOLDERS AND CONTRIBUTORS
% "AS IS" AND ANY EXPRESS OR IMPLIED WARRANTIES, INCLUDING, BUT NOT
% LIMITED TO, THE IMPLIED WARRANTIES OF MERCHANTABILITY AND FITNESS
% FOR A PARTICULAR PURPOSE ARE DISCLAIMED. IN NO EVENT SHALL THE
% COPYRIGHT OWNER OR CONTRIBUTORS BE LIABLE FOR ANY DIRECT, INDIRECT,
% INCIDENTAL, SPECIAL, EXEMPLARY, OR CONSEQUENTIAL DAMAGES (INCLUDING,
% BUT NOT LIMITED TO, PROCUREMENT OF SUBSTITUTE GOODS OR SERVICES;
% LOSS OF USE, DATA, OR PROFITS; OR BUSINESS INTERRUPTION) HOWEVER
% CAUSED AND ON ANY THEORY OF LIABILITY, WHETHER IN CONTRACT, STRICT
% LIABILITY, OR TORT (INCLUDING NEGLIGENCE OR OTHERWISE) ARISING IN
% ANY WAY OUT OF THE USE OF THIS SOFTWARE, EVEN IF ADVISED OF THE
% POSSIBILITY OF SUCH DAMAGE.
%%

%%
% This is the Eisvogel pandoc LaTeX template.
%
% For usage information and examples visit the official GitHub page:
% https://github.com/Wandmalfarbe/pandoc-latex-template
%%

% Options for packages loaded elsewhere
\PassOptionsToPackage{unicode}{hyperref}
\PassOptionsToPackage{hyphens}{url}
\PassOptionsToPackage{dvipsnames,svgnames,x11names,table}{xcolor}
%
\documentclass[
  paper=a4,
  ,captions=tableheading
]{scrartcl}
\usepackage{amsmath,amssymb}
% Use setspace anyway because we change the default line spacing.
% The spacing is changed early to affect the titlepage and the TOC.
\usepackage{setspace}
\setstretch{1.2}
\usepackage{iftex}
\ifPDFTeX
  \usepackage[T1]{fontenc}
  \usepackage[utf8]{inputenc}
  \usepackage{textcomp} % provide euro and other symbols
\else % if luatex or xetex
  \usepackage{unicode-math} % this also loads fontspec
  \defaultfontfeatures{Scale=MatchLowercase}
  \defaultfontfeatures[\rmfamily]{Ligatures=TeX,Scale=1}
\fi
\usepackage{lmodern}
\ifPDFTeX\else
  % xetex/luatex font selection
\fi
% Use upquote if available, for straight quotes in verbatim environments
\IfFileExists{upquote.sty}{\usepackage{upquote}}{}
\IfFileExists{microtype.sty}{% use microtype if available
  \usepackage[]{microtype}
  \UseMicrotypeSet[protrusion]{basicmath} % disable protrusion for tt fonts
}{}

\usepackage{pgfpages} 
\usepackage[export]{adjustbox}
\usepackage{graphicx}
\usepackage{ragged2e}
\makeatletter
\@ifundefined{KOMAClassName}{% if non-KOMA class
  \IfFileExists{parskip.sty}{%
    \usepackage{parskip}
  }{% else
    \setlength{\parindent}{0pt}
    \setlength{\parskip}{6pt plus 2pt minus 1pt}}
}{% if KOMA class
  \KOMAoptions{parskip=half}}
\makeatother
\usepackage{xcolor}
\definecolor{default-linkcolor}{HTML}{A50000}
\definecolor{default-filecolor}{HTML}{A50000}
\definecolor{default-citecolor}{HTML}{4077C0}
\definecolor{default-urlcolor}{HTML}{4077C0}
\usepackage[margin=2.5cm,includehead=true,includefoot=true,centering,]{geometry}
\usepackage[export]{adjustbox}
\usepackage{graphicx}
\usepackage{color}
\usepackage{fancyvrb}
\newcommand{\VerbBar}{|}
\newcommand{\VERB}{\Verb[commandchars=\\\{\}]}
\DefineVerbatimEnvironment{Highlighting}{Verbatim}{commandchars=\\\{\}}
% Add ',fontsize=\small' for more characters per line
\newenvironment{Shaded}{}{}
\newcommand{\AlertTok}[1]{\textcolor[rgb]{1.00,0.00,0.00}{\textbf{#1}}}
\newcommand{\AnnotationTok}[1]{\textcolor[rgb]{0.38,0.63,0.69}{\textbf{\textit{#1}}}}
\newcommand{\AttributeTok}[1]{\textcolor[rgb]{0.49,0.56,0.16}{#1}}
\newcommand{\BaseNTok}[1]{\textcolor[rgb]{0.25,0.63,0.44}{#1}}
\newcommand{\BuiltInTok}[1]{\textcolor[rgb]{0.00,0.50,0.00}{#1}}
\newcommand{\CharTok}[1]{\textcolor[rgb]{0.25,0.44,0.63}{#1}}
\newcommand{\CommentTok}[1]{\textcolor[rgb]{0.38,0.63,0.69}{\textit{#1}}}
\newcommand{\CommentVarTok}[1]{\textcolor[rgb]{0.38,0.63,0.69}{\textbf{\textit{#1}}}}
\newcommand{\ConstantTok}[1]{\textcolor[rgb]{0.53,0.00,0.00}{#1}}
\newcommand{\ControlFlowTok}[1]{\textcolor[rgb]{0.00,0.44,0.13}{\textbf{#1}}}
\newcommand{\DataTypeTok}[1]{\textcolor[rgb]{0.56,0.13,0.00}{#1}}
\newcommand{\DecValTok}[1]{\textcolor[rgb]{0.25,0.63,0.44}{#1}}
\newcommand{\DocumentationTok}[1]{\textcolor[rgb]{0.73,0.13,0.13}{\textit{#1}}}
\newcommand{\ErrorTok}[1]{\textcolor[rgb]{1.00,0.00,0.00}{\textbf{#1}}}
\newcommand{\ExtensionTok}[1]{#1}
\newcommand{\FloatTok}[1]{\textcolor[rgb]{0.25,0.63,0.44}{#1}}
\newcommand{\FunctionTok}[1]{\textcolor[rgb]{0.02,0.16,0.49}{#1}}
\newcommand{\ImportTok}[1]{\textcolor[rgb]{0.00,0.50,0.00}{\textbf{#1}}}
\newcommand{\InformationTok}[1]{\textcolor[rgb]{0.38,0.63,0.69}{\textbf{\textit{#1}}}}
\newcommand{\KeywordTok}[1]{\textcolor[rgb]{0.00,0.44,0.13}{\textbf{#1}}}
\newcommand{\NormalTok}[1]{#1}
\newcommand{\OperatorTok}[1]{\textcolor[rgb]{0.40,0.40,0.40}{#1}}
\newcommand{\OtherTok}[1]{\textcolor[rgb]{0.00,0.44,0.13}{#1}}
\newcommand{\PreprocessorTok}[1]{\textcolor[rgb]{0.74,0.48,0.00}{#1}}
\newcommand{\RegionMarkerTok}[1]{#1}
\newcommand{\SpecialCharTok}[1]{\textcolor[rgb]{0.25,0.44,0.63}{#1}}
\newcommand{\SpecialStringTok}[1]{\textcolor[rgb]{0.73,0.40,0.53}{#1}}
\newcommand{\StringTok}[1]{\textcolor[rgb]{0.25,0.44,0.63}{#1}}
\newcommand{\VariableTok}[1]{\textcolor[rgb]{0.10,0.09,0.49}{#1}}
\newcommand{\VerbatimStringTok}[1]{\textcolor[rgb]{0.25,0.44,0.63}{#1}}
\newcommand{\WarningTok}[1]{\textcolor[rgb]{0.38,0.63,0.69}{\textbf{\textit{#1}}}}

% Workaround/bugfix from jannick0.
% See https://github.com/jgm/pandoc/issues/4302#issuecomment-360669013)
% or https://github.com/Wandmalfarbe/pandoc-latex-template/issues/2
%
% Redefine the verbatim environment 'Highlighting' to break long lines (with
% the help of fvextra). Redefinition is necessary because it is unlikely that
% pandoc includes fvextra in the default template.
\usepackage{fvextra}
\DefineVerbatimEnvironment{Highlighting}{Verbatim}{breaklines,fontsize=\small,commandchars=\\\{\}}

\usepackage{longtable,booktabs,array}
\usepackage{calc} % for calculating minipage widths
% Correct order of tables after \paragraph or \subparagraph
\usepackage{etoolbox}
\makeatletter
\patchcmd\longtable{\par}{\if@noskipsec\mbox{}\fi\par}{}{}
\makeatother
% Allow footnotes in longtable head/foot
\IfFileExists{footnotehyper.sty}{\usepackage{footnotehyper}}{\usepackage{footnote}}
\makesavenoteenv{longtable}
% add backlinks to footnote references, cf. https://tex.stackexchange.com/questions/302266/make-footnote-clickable-both-ways
\usepackage{footnotebackref}
\setlength{\emergencystretch}{3em} % prevent overfull lines
\providecommand{\tightlist}{%
  \setlength{\itemsep}{0pt}\setlength{\parskip}{0pt}}
\setcounter{secnumdepth}{-\maxdimen} % remove section numbering
\ifLuaTeX
  \usepackage{selnolig}  % disable illegal ligatures
\fi
\IfFileExists{bookmark.sty}{\usepackage{bookmark}}{\usepackage{hyperref}}
\IfFileExists{xurl.sty}{\usepackage{xurl}}{} % add URL line breaks if available
\urlstyle{same}
\hypersetup{
  pdftitle={Nanomechanical Analysis of Tubular Cell Cytoskeleton},
  pdfauthor={Joseph Ashton},
  hidelinks,
  breaklinks=true,
  pdfcreator={LaTeX via pandoc with the Eisvogel template}}
\title{Nanomechanical Analysis of Tubular Cell Cytoskeleton}
\author{Joseph Ashton}
  \date{\today}




%%
%% added
%%


%
% for the background color of the title page
%
\usepackage{pagecolor}
\usepackage{afterpage}
\usepackage[margin=2.5cm,includehead=true,includefoot=true,centering]{geometry}

%
% break urls
%
\PassOptionsToPackage{hyphens}{url}

%
% When using babel or polyglossia with biblatex, loading csquotes is recommended
% to ensure that quoted texts are typeset according to the rules of your main language.
%
\usepackage{csquotes}

%
% captions
%
\definecolor{caption-color}{HTML}{777777}
\usepackage[font={stretch=1.2}, textfont={color=caption-color}, position=top, skip=4mm, labelfont=bf, singlelinecheck=false, justification=justified]{caption}
\setcapindent{0em}

%
% blockquote
%
\definecolor{blockquote-border}{RGB}{221,221,221}
\definecolor{blockquote-text}{RGB}{119,119,119}
\usepackage{mdframed}
\newmdenv[rightline=false,bottomline=false,topline=false,linewidth=3pt,linecolor=blockquote-border,skipabove=\parskip]{customblockquote}
\renewenvironment{quote}{\begin{customblockquote}\list{}{\rightmargin=0em\leftmargin=0em}%
\item\relax\color{blockquote-text}\ignorespaces}{\unskip\unskip\endlist\end{customblockquote}}

%
% Source Sans Pro as the de­fault font fam­ily
% Source Code Pro for monospace text
%
% 'default' option sets the default
% font family to Source Sans Pro, not \sfdefault.
%
\ifnum 0\ifxetex 1\fi\ifluatex 1\fi=0 % if pdftex
    \usepackage[default]{sourcesanspro}
  \usepackage{sourcecodepro}
  \else % if not pdftex
    \usepackage[default]{sourcesanspro}
  \usepackage{sourcecodepro}

  % XeLaTeX specific adjustments for straight quotes: https://tex.stackexchange.com/a/354887
  % This issue is already fixed (see https://github.com/silkeh/latex-sourcecodepro/pull/5) but the
  % fix is still unreleased.
  % TODO: Remove this workaround when the new version of sourcecodepro is released on CTAN.
  \ifxetex
    \makeatletter
    \defaultfontfeatures[\ttfamily]
      { Numbers   = \sourcecodepro@figurestyle,
        Scale     = \SourceCodePro@scale,
        Extension = .otf }
    \setmonofont
      [ UprightFont    = *-\sourcecodepro@regstyle,
        ItalicFont     = *-\sourcecodepro@regstyle It,
        BoldFont       = *-\sourcecodepro@boldstyle,
        BoldItalicFont = *-\sourcecodepro@boldstyle It ]
      {SourceCodePro}
    \makeatother
  \fi
  \fi

%
% heading color
%
\definecolor{heading-color}{RGB}{40,40,40}
\addtokomafont{section}{\color{heading-color}}
% When using the classes report, scrreprt, book,
% scrbook or memoir, uncomment the following line.
%\addtokomafont{chapter}{\color{heading-color}}

%
% variables for title, author and date
%
\usepackage{titling}
\title{Nanomechanical Analysis of Tubular Cell Cytoskeleton}
\author{Joseph Ashton}
\date{}

%
% tables
%

\definecolor{table-row-color}{HTML}{F5F5F5}
\definecolor{table-rule-color}{HTML}{999999}

%\arrayrulecolor{black!40}
\arrayrulecolor{table-rule-color}     % color of \toprule, \midrule, \bottomrule
\setlength\heavyrulewidth{0.3ex}      % thickness of \toprule, \bottomrule
\renewcommand{\arraystretch}{1.3}     % spacing (padding)


%
% remove paragraph indentation
%
\setlength{\parindent}{0pt}
\setlength{\parskip}{6pt plus 2pt minus 1pt}
\setlength{\emergencystretch}{3em}  % prevent overfull lines

%
%
% Listings
%
%


%
% header and footer
%
\usepackage[headsepline,footsepline]{scrlayer-scrpage}

\newpairofpagestyles{eisvogel-header-footer}{
  \clearpairofpagestyles
  \ihead*{Nanomechanical Analysis of Tubular Cell Cytoskeleton}
  \chead*{}
  \ohead*{}
  \ifoot*{Joseph Ashton}
  \cfoot*{}
  \ofoot*{\thepage}
  \addtokomafont{pageheadfoot}{\upshape}
}
\pagestyle{eisvogel-header-footer}



%%
%% end added
%%
\usepackage{pgfpages}
\usepackage[export]{adjustbox}
\usepackage{graphicx}
\usepackage{ragged2e}


\begin{document}

%%
%% begin titlepage
%%
\begin{titlepage}
\newcommand{\colorRule}[3][black]{\textcolor[HTML]{#1}{\rule{#2}{#3}}}
\begin{flushleft}
\noindent
\\[-1em]
\color[HTML]{000000}
\makebox[0pt][l]{\colorRule[FFFFFF]{1.3\textwidth}{0pt}}
\par
\noindent

{
  \begin{center}
  \setstretch{1.4}
  \vfill
  \noindent {\huge \textbf{\textsf{
      Nanomechanical Analysis of Tubular Cell Cytoskeleton
  }}}
    \vskip 2em
  \noindent {\Large \textit{Joseph Ashton}}
  \vfill
  \end{center}
}

\noindent
\begin{center}
\includegraphics[width=35mm]{/home/joeashton/Sync/Obsidian/SuperVault/Core/Templates/Pandoc/attachments/UoL\_logo.png}
\end{center}
\end{flushleft}
\end{titlepage}
\restoregeometry
\pagenumbering{arabic} 

%%
%% end titlepage
%%

% \maketitle
\pagenumbering{Roman} % set the numbering style to lowercase letter

\begin{center}
 {\LARGE \textbf{\textsf{Abstract}}}
\end{center}

\begin{abstract}
\begin{justify}
    This project investigates changes in mechanical properties of kidney
    cells when exposed to TGF-β1, which is known to induce renal disease
    {[}@gentleME2013-EpithelialCellTGFv{]}. The aim of this project is
    to provide insight on the progression of diabetic nephropathy from a
    mechanical perspective based on changes in mechanical properties
    observed in single cells using atomic force microscopy.
  \end{justify}
\end{abstract}
\pagebreak


\begin{center}
 {\LARGE \textbf{\textsf{Acknowledgements}}}
\end{center}

\begin{abstract}
\begin{justify}
I would like to thank my patient amd knowledgable suporvisor Eleftherios
Siamantouras
\end{justify}
\end{abstract}
\pagebreak


\pagenumbering{arabic} % set the numbering style to lowercase letter
\setcounter{page}{0} % Set the page counter to 3



\renewcommand*\contentsname{}
\renewcommand*\contentsname{Table of Contents}
{
\setcounter{tocdepth}{3}
\tableofcontents
\newpage
}
\subsection{Abstract}\label{abstract}

This project investigates changes in mechanical properties of kidney
cells when exposed to TGF-β1, which is known to induce renal disease
{[}@gentleME2013-EpithelialCellTGFv{]}. The aim of this project is to
develop a method to quantify the progression of diabetic nephropathy
from a mechanical perspective based on changes in mechanical properties
observed in single cells using atomic force microscopy.

\subsection{Introduction}\label{introduction}

The mechanical properties of cells are finely tuned to their function,
especially epithelial cells who's core role is to form active structural
surfaces where correct functioning is a direct result of appropriate
strength, stiffness, and shape. In patients suffering from diabetic
nephropathy the alterations in renal tubule cell properties are directly
associated with progression of the disease and further kidney damage.

\subsection{Background}\label{background}

\subsubsection{Relevant Physiology}\label{relevant-physiology}

\%\% \#\#\#\# Kidney \%\%

The human body can be understood as a complex biological machine, made
up of many sub-mechanisms familiar to engineers. In this sense the
filtration system of the human body is referred to as the renal system,
in which the kidneys are a component about the size of a clenched fist
that can be likened to a sophisticated water treatment plant combined
with a feedback-controlled chemical processing unit. Each contain
roughly a million multi step filter loops called nephrons
{[}@bertramJF2011-HumanNephronNumber{]}.

\begin{quote}
{} Labeled Kidney and Nephron form National Institute of Diabetes and
Digestive and Kidney Diseases, National Institutes of Health
{[}@niddk-KidneyNephronLabeled{]}.
\end{quote}

\%\% \#\#\#\# Nephrons \%\%

The nephrons are selective, able to remove waste products while keeping
desirable substances in the blood. They are able to regulating essential
substances such as water, electrolytes, and pH levels to strict set
points. {[}@ogobuiroI2025-PhysiologyRenal{]}

\%\% \#\#\#\# glomerulus \%\%

The the first step unfiltered blood enters the glomerulus and if forced
through several membrane filters by hydro-static pressure. The first
layer permits all solutes blocking only cells. The next is negatively
charged thus blocking proteins like albumin. The final layer modulates
the flow resistance to vary the hydro-static pressure gradient, this
will be counter balanced by the osmotic pressure such that it can be
used to effectively vary the ultra filtration coefficient.
{[}@pavenstadtH2000-RolesPodocyteGlomerular;
@ogobuiroI2025-PhysiologyRenal{]} Leaving the glomerulus is a blood
vessel containing only cells and proteins and a fractional remainder of
the other solute, and the tubule carrying all the removed solute
{[}@lumen-NephronStructure{]}.

\%\% \#\#\#\# Tubule \%\%

The glomerulus is an overly aggressive filter; much of the water and
solute must be re introduced to the blood from the tubules. \%\%in
several stages. The first stage, the proximal convoluted tubule (PCT)
pumps all the glucose and amino acids as well as most of the sodium and
water back into the blood {[}@ogobuiroI2025-PhysiologyRenal{]}. In the
next more water is reabsorbed, and in the final the remainder of the
sodium as well as potassium and chlorides get reabsorbed via osmosis
{[}@ogobuiroI2025-PhysiologyRenal{]}. \%\% The tubules run along side
blood vessels and using a combination of osmosis, active transport and
controlled ionic gradients \%\%sodium, potassium, calcium, magnesium,
chlorides\%\% the valuable ions and most of the water is reabsorbed over
several uniquely specialised segments
{[}@ogobuiroI2025-PhysiologyRenal{]}.

\%\% \#\#\#\# Tubule cell \%\%

\begin{quote}
{} Simplified diagram of tubule, tubule wall and tubule cell structure.
The structure of the tubule varies significantly across it's length to
as different sections are specialised to permeate different resources,
the lumen diameter and epithelial cell height values are averages of
random samples {[}@morozovD2021-MappingKidneyTubule{]}.
\end{quote}

\%\% Breakdown of the layers and cell types \#WIP \%\%

Epithelial tubule cells are the essential building blocks of tubules.
They are anchored to each other and to the extra cellular matrix (ECM)
by junctions tied to their cytoskeleton
{[}@yuASL2013-Chapter12Intercellular; @zihniC2016-TightJunctions{]}. In
this way the cytoskeleton plays an essential role in maintaining the
structure of both the individual cells and the larger structure.

\%\% cytoskeleton \%\%

\subsubsection{Diabetic Nephropathy (DN)}\label{diabetic-nephropathy-dn}

\%\% \#\#\#\# Summary \%\% Diabetic nephropathy (DN) is a common and
serious complication of diabetes resulting in kidney failure due to
progressive damage to the nephrons, the functional units of the kidney
responsible for filtering the blood.

\%\% \#\#\#\# Epidemiology \& Prognosis \%\% Diabetic nephropathy
develops in 30-40\% of people with diabetes after 15-20 years, as the
disease progresses the damage accumulates and mortality rate rises
{[}@vargheseRT2025-DiabeticNephropathy{]}. Based on the risk factor of
the patient treatments range from lifestyle changes and medications, to
renal replacement which involves dialysis or transplantation
{[}@vargheseRT2025-DiabeticNephropathy{]}.

\%\% Stats about the number of people with it and dying of it and how
that is changing over time \%\%

\%\% connecting line from mortality to pathophysiology \%\%

\%\% \#\#\#\# Relevant Pathophysiology \%\% In type 1 diabetes a lack of
insulin and in type 2 Insulin resistance cause chronic hyperglycemia a
condition where there is too much glucose in the blood. Hyperglycemia
causes an increased build up of reactive oxygen species (ROS) this
ongoing oxidative stress causes chronic inflammation
{[}@gonzalezP2023-HyperglycemiaOxidativeStress{]}.

Inflammation increases production of cytokines, including TGF-β1, which
trigger Epithelial to Mesenchymal Transition (EMT)
{[}@hillsCE2012-TGFvModulatesCelltocell;
@pizzinoG2017-OxidativeStress{]}. EMT is a process where cells which
make up structural and functional surfaces (epithelial) transition into
repair/maintenance cells (mesenchymal)
{[}@kalluriR2009-BasicsEpithelialmesenchymalTransition{]}. In this case
tubular epithelial, cells which make up the fine vessels of the kidney
that filter blood, transform into myofibroblasts, repair and maintenance
cells \%\% or undergo apoptosis (cell death) \%\%
{[}@iwanoM2002-EvidenceThatFibroblasts{]}. This is the underlying
mechanism of fibrosis, which induces atrophy and scarring in the tubules
\%\% and results in intraglomerular hypertension \%\% causing
progressive kidney damage
{[}@metcalfeW2007-HowDoesEarlyChronicKidneyDiseaseProgress{]}.

\subsubsection{Atomic Force Microscopy
(AFM)}\label{atomic-force-microscopy-afm}

\%\% \#\#\#\# What is AFM? \%\%

Atomic Force Microscopy (AFM) is a technique for characterising
nanomechanical properties and structure. It is well suited to
microbiology as it allows for the study of live cells
{[}@kilpatrickJI2015-NanomechanicsCellsBiomaterials{]}.

Atomic force microscopes use the deflection of a very fine probe on a
flexible cantilever to detect contact forces ranging form nano to micro
Newtons. There are a myriad of applications and operating modes of AFM
{[}@dufreneYF2002-AtomicForceMicroscopy{]} but this report is primarily
concerned with nano indentation. This involves advancing a fine tipped
probe on the end of the cantilever into a sample cell producing a force
over indentation depth curve, from which the elasticity of the cell can
be calculated using a Hertz contact model
{[}@dufreneYF2002-AtomicForceMicroscopy;
@buttHJ1995-MeasuringSurfaceForces{]}.

\%\% \#\#\#\# functional/mechanical description \%\%

The typically atomic force microscope utilise a laser focused on the
free end of the cantilever such that any deflection of the probe
produces an amplified deflection of the reflected beam, this is recorded
by a position sensitive photodiode
{[}@dufreneYF2002-AtomicForceMicroscopy;
@buttHJ1995-MeasuringSurfaceForces{]}.

\begin{quote}
{} Atomic force microscope functional diagram
\end{quote}

The sample once mounted to the sample stage can be manoeuvred precisely
in relative to the probe by applying voltage to piezoelectric actuators
{[}@dufreneYF2002-AtomicForceMicroscopy;
@buttHJ1995-MeasuringSurfaceForces;
@giraudF2019-PiezoelectricActuatorsIntroduction{]} this is how the
sample is advanced into the tip. Once calibrated the voltage at the
actuators gives the sample stage position and the voltage at the
photodiode gives the deflection of the probe, with this a force
displacement curve can be produced by accounting for the stiffness of
the cantilever and the relative displacement
{[}@dufreneYF2002-AtomicForceMicroscopy;
@buttHJ1995-MeasuringSurfaceForces;
@kilpatrickJI2015-NanomechanicsCellsBiomaterials{]}.

\%\% \#\#\#\# typical force displacement curve \%\%

A typical force displacement curve from a nano indention experiment has
the following shape seen in figure \%\% \#WIP \%\% (A) below. A broadly
level region where the probe is not in contact with the cell; the
contact region; a sloped region where the probe is indenting the cell;
the turnaround point; from which the same is repeated in reverse
differing mainly at the point of separation
{[}@kilpatrickJI2015-NanomechanicsCellsBiomaterials{]}.

\begin{quote}
{} {} Example AFM data from Radmacher 2007, (A) shows the curve as a
whole, (B) zoomed into the contact / separation region
{[}@radmacherM2007-StudyingMechanicsCellular{]}.
\end{quote}

\%\% Contact point jump \%\%

The exact point of contact is often ambiguous and rarely the same as the
the point of separation. On approach the cantilever will be deflected
away from the cell by van der waals forces until the spring force of the
cantilever overcomes and surface tension takes hold
{[}@dufreneYF2002-AtomicForceMicroscopy;
@buttHJ1995-MeasuringSurfaceForces;
@kilpatrickJI2015-NanomechanicsCellsBiomaterials{]}. The point of
separation is typically clearer as it's associated with a ``jump'' in
cantilever deflection as the surface tension / adhesion of the cell to
the probe is overcome {[}@dufreneYF2002-AtomicForceMicroscopy;
@buttHJ1995-MeasuringSurfaceForces;
@kilpatrickJI2015-NanomechanicsCellsBiomaterials{]}.

\%\% \textgreater{} {} \textgreater{}
{[}@buttHJ1995-MeasuringSurfaceForces{]} \%\%

\subsubsection{Contact Mechanics}\label{contact-mechanics}

In order to calculate elasticity the experimental data must be fit to a
theoretical mechanical model of the interaction. Below is a table
outlining different model indention relationships.

\begin{longtable}[]{@{}
  >{\raggedright\arraybackslash}p{(\linewidth - 4\tabcolsep) * \real{0.0604}}
  >{\raggedright\arraybackslash}p{(\linewidth - 4\tabcolsep) * \real{0.2458}}
  >{\raggedright\arraybackslash}p{(\linewidth - 4\tabcolsep) * \real{0.6938}}@{}}
\toprule\noalign{}
\begin{minipage}[b]{\linewidth}\raggedright
Model
\end{minipage} & \begin{minipage}[b]{\linewidth}\raggedright
Force-Indentation relationship
\end{minipage} & \begin{minipage}[b]{\linewidth}\raggedright
Scope
\end{minipage} \\
\midrule\noalign{}
\endhead
\bottomrule\noalign{}
\endlastfoot
Hertz & \(F = \frac{4}{3}E' \sqrt{R} \ \omega^{3/2}\) & Hertz model
approximates the shallow indention of two linearly elastic spheres with
infinitesimal strains
{[}@linDC2009-SphericalIndentationSoftMatterHertzianRegime;
@radmacherM2007-StudyingMechanicsCellular;
@jpkinstruments-JPKDataProcessing{]}. \\
DMT (DerjaguinMuller-Toporov) &
\(F = F_{Hertz} - F_{det}\)\(\delta = \frac{a}{2} \ln \frac{R_{i}+a}{R_{i}-a}\)
& Depending on the depth of indentation and the material interaction it
can be important to account electrostatic non contact forces, the
influence of which can be modelled using the Derjaguin approximation for
interaction potential {[}@buttHJ1995-MeasuringSurfaceForces;
@jpkinstruments-JPKDataProcessing{]}. \\
Fung &
\(F = B\pi (\frac{a^5- 15Ra^4 + 75R^2a^3}{5Ra^2- 50R^2a + 125R^3})\text{exp}⁡[b(\frac{a^3- 15Ra^2}{25R^2a- 125R^3})]\)
& An exponential strain energy function based on mechanical testing of
mesentery and arterial tissues, that models the non linear elasticity of
cells {[}@fungY1967-ElasticitySoftTissues;
@linDC2009-SphericalIndentationSoftMatterHertzianRegime{]}. This method
is tangebly more precise but doesn't provide a simple value for young's
modulus. \\
\end{longtable}

\%\% Expand on obstacles with applying contact mechanics models to cells
\%\%

\subsection{Literature Review}\label{literature-review}

Developments in both understanding \%\% of the pathophysiology \%\% of
kidney disease and the application of atomic force microscopy (AFM)
technology {[}@liuS2024-AtomicForceMicroscopyDisease-relatedStudies;
@dufreneYF2002-AtomicForceMicroscopy{]} may provide a valuable measure
of the progression of kidney failure to inform the research and
development of novel therapies
{[}@parrishAR2016-CytoskeletonNovelTarget{]}.

\%\% \#\#\#\# What we measure \%\%

\%\% Single cell stiffness \%\% The mechanical properties of tubular
cells are largely a result of their cytoskeletal structure
{[}@jalilianI2015-CellElasticityRegulated;
@radmacherM2007-StudyingMechanicsCellular{]} which is altered
significantly with the progression of DN
{[}@buckleyST2012-CytoskeletalRearrangementTGFv1induced{]}.

\%\% Inter cellular adhesion \%\%

\%\% The stiffness of individual cells can be observed and correlated
with the {[}@siamantourasE2016-QuantifyingCellularMechanics{]}. \%\%

\%\% \#\#\#\# Obstacles \& limitations \%\%

\%\% Cell cultures aren't perfect representations of their in vitro
counterparts. This can be improved with careful preparation. The
elasticity of the substrate the culture is grown on can have a
significant impact on cytoskelital arrangement
{[}@wangD2022-KidneyProximalTubule;
@loveH2018-SubstrateElasticityGoverns{]} \%\%

\%\% Table with papers and a relevant summary \#WIP \%\%

The below table lists several papers utilising atomic force microscopes
to produce force displacement curves from a bead tipped cantilever
fitted to a hertz contact model to find cell elasticity.

\begin{longtable}[]{@{}
  >{\raggedright\arraybackslash}p{(\linewidth - 6\tabcolsep) * \real{0.5803}}
  >{\raggedright\arraybackslash}p{(\linewidth - 6\tabcolsep) * \real{0.1116}}
  >{\raggedright\arraybackslash}p{(\linewidth - 6\tabcolsep) * \real{0.2218}}
  >{\raggedright\arraybackslash}p{(\linewidth - 6\tabcolsep) * \real{0.0863}}@{}}
\toprule\noalign{}
\begin{minipage}[b]{\linewidth}\raggedright
Paper
\end{minipage} & \begin{minipage}[b]{\linewidth}\raggedright
Cell Type
\end{minipage} & \begin{minipage}[b]{\linewidth}\raggedright
Scope
\end{minipage} & \begin{minipage}[b]{\linewidth}\raggedright
Cell Elasticity
\end{minipage} \\
\midrule\noalign{}
\endhead
\bottomrule\noalign{}
\endlastfoot
{[}@siamantourasE2016-QuantifyingCellularMechanics{]} E. Siamantouras,
C. E. Hills, P. E. Squires, and K.-K. Liu, `Quantifying cellular
mechanics and adhesion in renal tubular injury using single cell force
spectroscopy', \emph{Nanomedicine: Nanotechnology, Biology and
Medicine}, vol.~12, no. 4, pp.~1013--1021, May 2016, doi:
\href{https://doi.org/10.1016/j.nano.2015.12.362}{10.1016/j.nano.2015.12.362}.
& HK2: immortalised human kidney proximal tubule epithelial cell culture
& Over 30 cells each indented 5 times immediately above the nucleus
producing over 150 curves. & \textbf{control}: 320Pa cells treated with
TGF-β1: 549 Pa \\
{[}@jafariA2024-MechanicalPropertiesHuman{]} A. Jafari, A. Sadeghi, and
M. Lafouti, `Mechanical properties of human kidney cells and their
effects on the atomic force microscope beam vibrations', \emph{Microsc.
Res. Tech.}, vol.~87, no. 8, pp.~1704--1717, 2024, doi:
\href{https://doi.org/10.1002/jemt.24543}{10.1002/jemt.24543}. &
HEK-293: immortalised human embryonic kidney cell culture & did not
elaborate & 539.8 Pa \\
{[}@shimizuY2012-SimpleDisplaySystem{]} Y. Shimizu, T. Kihara, S. M. A.
Haghparast, S. Yuba, and J. Miyake, `Simple Display System of Mechanical
Properties of Cells and Their Dispersion', \emph{PLOS ONE}, vol.~7, no.
3, p.~e34305, Mar.~2012, doi:
\href{https://doi.org/10.1371/journal.pone.0034305}{10.1371/journal.pone.0034305}.
& HEK-293: immortalised human embryonic kidney cell culture & The median
of value of over 100 cells examined at 25 points each. & mode value
(\(x_{0}\)): 410 Pa variance (\(w\)): 0.757 \\
{[}@buckleyST2012-CytoskeletalRearrangementTGFv1induced{]} S. T.
Buckley, C. Medina, A. M. Davies, and C. Ehrhardt, `Cytoskeletal
re-arrangement in TGF-β1-induced alveolar epithelial-mesenchymal
transition studied by atomic force microscopy and high-content
analysis', \emph{Nanomedicine: Nanotechnology, Biology and Medicine},
vol.~8, no. 3, pp.~355--364, Apr.~2012, doi:
\href{https://doi.org/10.1016/j.nano.2011.06.021}{10.1016/j.nano.2011.06.021}.
& A549: human lung alveolar carcinoma epithelial cell culture & On each
cell, a 4 × 4 grid of force-distance curves was collected in at least 5
different positions (avoiding the nucleus and the very edge) producing
over 750 curves. & On Glass: 8300 \(\pm\) 1100 PaOn collagen I: 9100
\(\pm\) 2900 Pa \\
{[}@wyssHM2011-BiophysicalPropertiesNormal{]} H. M. Wyss \emph{et al.},
`Biophysical properties of normal and diseased renal glomeruli',
\emph{Am J Physiol Cell Physiol}, vol.~300, no. 3, pp.~C397--C405,
Mar.~2011, doi:
\href{https://doi.org/10.1152/ajpcell.00438.2010}{10.1152/ajpcell.00438.2010}.
& Sprague-Dawley rat kidney glomeruli capillary wall extracted by
differential sieving & 10 different glomeruli with 10 measurements each
& 2,300 \(\pm\) 160 Pa \\
\end{longtable}

\subsection{Methodology}\label{methodology}

The experimental procedure used to produce the data used in this report
is detailed thoroughly in reference
{[}@siamantourasE2016-QuantifyingCellularMechanics{]}. Cells of the
adult human proximal tubule kidney (HK2) cell line
{[}@ryanMJ1994-HK-2{]} where purchased from the American Type Culture
Collection (ATCC; Gaithersburg, MD 20878 USA). These where maintained in
Dulbecco's Modified Eagle Medium/Nutrient Mixture F-12 (DMEM/F12), 10\%
fetal calf serum (FCS), glutamine (2mM), and EGF (5ng/ml) for 48 hours.
The cells where divided into 2 test groups, the ``control'' group and
the ``treated'' group which where where serum starved overnight before
being exposed to TBF-β, an E-cadherin antibody obtained from R\&D
systems, at (2-10ng/mL) for a further 48 hours.

The indentation experiments where carried out using a JPK Instruments
CellHesion©200 module with a BioCell™ temperature controller to maintain
a bed temperature of 37°C on a TMC 63-530 anti-vibration table. Probes
where constructed by attaching 11µm Polyscience PolyBeads® to Nanoworld
TL-1 tipless cantilevers with a force constant of 0.03N/m. Each cell was
indented 5 times directly above the nucleus at a constant speed of 5µm/s
with intervals of 60 seconds. For each set of experiments the spring
constant of the cantilever was calibrated using the thermal noise method
and the cell's height was measured to determine an appropriate
indentation depth to minimise the influence of the hard basal substrate.

\%\% thermal noise method how and why \%\%

\%\% on a Nikon eclipse TE 300 inverted microscope. The AFM bed was
maintained at 37°C using a JPK Instruments BioCell™ temperature
controller. \%\%

\%\% How did i get ym values from the datasets \%\%

Experimental data was received in the form of \texttt{.jpk-force} logs
of head height position against vertical deflection force along with
experimental metadata. The JPK data processing software was used to
calculate the probe height based on spring constant and at this point
the curves where exported in text form. In order to establish
``trustworthy'' values for YM the function included in the JPK data
processing software was used with the deepest point of indentation to
1µm past the contact point as upper and lower bounds. This was then
replicated for the text exports in python using \texttt{nanite} an open
source package that offers the same Hertz/Sneddon elasticity model
truncated power series approximation for spherical indenters with a
difference in fit optimisation methods; where JPK Data Processing uses
least squared regression, Nanite utilises machine learning for fit
quality estimation and optimisation. Despite not providing the
``trustworthy'' fits as a rated training dataset Nanite reproduced the
``trustworthy'' YM estimations with an average deviation of less than
±0.05\%. The Hertz parabolic indenter model was also tested and compared
with the Hertz/Sneddon approximation. All force indentation curves where
plotted alongside those implied by the fitting along with the residual
fitting error to identify potential anomalies or systematic error.
Attention was paid to identify any consistent trends in the residual as
If the residual where to consistently deviate from generally flat noise
at 0 this would imply a poorly matched elasticity model.

\%\% How did I determine the appropriate cell YM given the experimental
data \%\%

As each cell was tested 5 times the apparent YM of a cell was taken to
be the average average of the Hertz/Sneddon fits. To validate the
results the force indentation curves of the experiments and the implied
curve of the apparent YM where plotted and inspected visually checking
for cell relaxation or systematic error based on observable trends in
successive experimentation or any apparent anomalies. In addition the
95\% confidence interval of the apparent cell YM was calculated for the
natural set and a \(100 \times\) bootstrapped super set this was
inspected on log scales given that the error margin increases
proportionally with higher YM, however for consistency with the rest of
the report figures included in the results section maintain linear YM
axis of 0,2000.

\%\% Confidence Intervals, what are they and how did I calculate them
\%\%

As the apparent cell YM was taken to be the mean it's confidence
intervals where those of the mean YM for the set of cell tests.

\[\text{CI}_\mu = \left[ \mu - t^* \cdot \frac{\sigma}{\sqrt{n}},\ \mu + t^* \cdot \frac{\sigma}{\sqrt{n}} \right]
\qquad
\text{CI}_\sigma = \left[ \sqrt{ \frac{(n-1) \cdot \sigma^2}{\chi^2_{\text{upper}}} },\ \sqrt{ \frac{(n-1) \cdot \sigma^2}{\chi^2_{\text{lower}}} } \right]
\]

Where confidence intervals where calculated for the standard deviation
as is later necessary in determining the confidence in the group
classifications and when montecarlo sampling, the chi distribution is
used. This was originally tried using normal distributions being a
generally acceptable approximation, however given the small and bottom
biased experiment sample sets assuming symmetric distribution of
probable standard deviations was not a fair representation.

\%\% Bootstrapping, what is it, why did I do it \%\%

A Bayes classifier was constructed to quantify the probability of
diabetic nephropathy from cell stiffness based on the effect observed in
the experimental data. The control group is taken as a model of healthy
cell presentation and the treated group representing the onset of
diabetic nephropathy. Similarly to how cell properties where estimated
from several tests, the typical group properties are estimated from
several cells, conversely it can also be found by taking the averages
and standard deviations of the whole dataset. It is often the case that
considering the whole raw dataset provides more accurate picture of the
group, however in this case it is appropriate to consider by subgroups
i.e.~by cells, this is because the samples are not independent and not
representative of the test case. As it has been observed that successive
tests are not introducing systematic error their average provides a more
accurate estimation of the given cell, thus classification should be
considered at the cell level.

\%\% The following is the general formula for normal distribution:

\[
{\large f(x)=
\frac{1}{\sigma \sqrt{2 \pi}}
e^{\frac{-1}{2}
\left( \cfrac{x-\mu}{\sigma}\right)^{2}}
}
\qquad
\begin{align}
f(x)    &=  \text{Probability Density Function}\\
x       &=  \text{Variable (i.e. Young's Modulus)}\\
\sigma  &=  \text{Standard Deviation}\\
\mu     &=  \text{Mean}\\
\end{align}
\] \%\%

\%\% Bayes Classifier \%\%

Bayes Theorem (Eq below) enables us to quantify the probability a cell
is diseased given its YM by considering the posterior probability that
it is an occurrence in a group with the appropriate probability density
function.

\[\large
P(G∣x) = \frac{P(x∣G) \cdot P(G)}{P(x)} 
\qquad
\begin{align}
P(G∣x)  &=  \text{Posterior Probability}\\
P(x∣G)  &=  \text{Likleyhood}\\
P(G)    &=  \text{Prior Probability}\\
P(x)    &=  \text{Evidence}\\
\end{align}
​\]

If we take healthy and diseased to be exclusive groups \(G_{1}\) and
\(G_{2}\) then the probability of a cell being diseased would be given
by:

\[\large P(G_2 \mid x) = \frac{P(x \mid G_2) \cdot P(G_2)}{P(x \mid G_1) \cdot P(G_1) + P(x \mid G_2) \cdot P(G_2)}\]

Where the likelihood of a given group is based on the normal
distribution implied by the mean and standard deviation of the YM
observed in the experimental data.

\%\%
\[P(x \mid G) = \frac{1}{\sqrt{2\pi}\sigma_G} \exp\left( -\frac{(x - \mu_G)^2}{2\sigma_G^2} \right)
\] \%\%

\[
{\large P(x \mid G) = 
\frac{1}{\sigma_{G} \sqrt{2 \pi}}
e^{\tfrac{-1}{2}
\left( \cfrac{x-\mu_{G}}{\sigma_{G}}\right)^{2}}
}
\qquad
\begin{align}
P(x \mid G) &=  \text{Group Probability Density Function}\\
x           &=  \text{Variable (i.e. Young's Modulus)}\\
\sigma_{G}  &=  \text{Group Standard Deviation}\\
\mu_{G}     &=  \text{Group Mean}\\
\end{align}
\]

And the prior probabilities will depend on the application, for high
throughput screening this wold be heavily biased towards the initial
cell state, or in patient diagnosis this could be a function of patient
specific and/or epidemiological factors. In the context of this report
prior probabilities are simply the proportion of samples from each
group.

\subsection{Results}\label{results}

The difference between the Hertz elasticity model for a parabolic
indentation and the Hertz/Sneddon spherical indentation where minor
producing effectively indistinguishable estimates for YM however the
Hertz parabolic model resulted in a slightly but consistently higher
residual fit error with it's slightly more progressive curvature. Below
are representative examples of each fit for the same force indentation
curve.

\%\% Quantify difference in Hertz v Sneddon average residuals \%\%

Figure: Comparison in Elasticity Fit Techniques for an Example Curve

--- start-multi-column: ID\_31no

\begin{Shaded}
\begin{Highlighting}[]
\NormalTok{Number of Columns: 2}
\NormalTok{Largest Column: standard}
\end{Highlighting}
\end{Shaded}

{}

--- column-break ---

{}

--- end-multi-column

There where some trends observed in the general shape of the residual
error, specifically; 1) an initial hump at the contact point likely due
to unaccounted for electrostatic non contact forces due to the van der
walls effect, 2) a middle dip and final flick where fit's are shallower
than the actual force indention behaviour implying an under estimation
of YM or a non linear elasticity. Both of these effects are particularly
pronounced in the following fitting for this dataset this would be
considered a bad fit.

{}

\%\% Figure showing all residual curves faintly with smoothed average of
all and grouped \%\%

\%\% Checking for cell relaxation or systematic error based on observed
trends in successive experimentation \%\%

There where no general trends observed across successive tests meaning
there was no need to control for cell relaxation. The majority of cells
showed strong agreement across tests resulting in tight confidence
intervals and representative apparent YM values. The examples below are
emblematic of typical samples from each group.

Figure: Example Cell Fittings for Control vs Treated Group

--- start-multi-column: ID\_xdg9

\begin{Shaded}
\begin{Highlighting}[]
\NormalTok{Number of Columns: 2}
\NormalTok{Largest Column: standard}
\end{Highlighting}
\end{Shaded}

{}

--- column-break ---

{}

--- end-multi-column

There where cells that displayed significantly higher variation between
experiments from both groups however this was not constantly associated
with the order of the tests. Given the shallow depth of the indention
this is not likely the influence of stiffer organelles but could perhaps
be due to the probing site interacting with cytoskelital structures such
as the microvilli force sensing/transducing elements or structural
anchor points, however It would require more advanced imaging techniques
to explain these variations with confidence. Notably cells in the
treated group tended to have one test with a significantly lower
apparent elasticity but strong agreement in the other 4 as is the case
below.

--- start-multi-column: ID\_2ugn

\begin{Shaded}
\begin{Highlighting}[]
\NormalTok{Number of Columns: 2}
\NormalTok{Largest Column: standard}
\end{Highlighting}
\end{Shaded}

{}

--- column-break ---

{}

--- end-multi-column

\%\% \#\#\# Validity of results \%\%

The treated cells where found to be on average twice as stiff as the
untreated cells with a 500 Pa higher average young's modulus in addition
the treated group shows a 77\% higher variance with a standard deviation
of 541 Pa.

--- start-multi-column: ID\_3r15

\begin{Shaded}
\begin{Highlighting}[]
\NormalTok{Number of Columns: 2}
\NormalTok{Largest Column: standard}
\end{Highlighting}
\end{Shaded}

Figure: Cell Young's Modulus by test group

{}

Table: Cell Young's Modulus (Pa) statistics by test group

\begin{longtable}[]{@{}lllllll@{}}
\toprule\noalign{}
Group & Mode & Min & Max & Median & Mean & StDev \\
\midrule\noalign{}
\endhead
\bottomrule\noalign{}
\endlastfoot
Control & 154.96 & 143.85 & 982.09 & 392.04 & 457.99 & 305.52 \\
Treated & 524.65 & 524.65 & 1761.58 & 807.94 & 975.53 & 540.96 \\
\end{longtable}

--- column-break ---

Figure: Experiment-wise Young's Modulus (Pa) by test group

{}

Table: Experiment-wise Young's Modulus (Pa) statistics by test group

\begin{longtable}[]{@{}lllllll@{}}
\toprule\noalign{}
Group & Mode & Min & Max & Median & Mean & StDev \\
\midrule\noalign{}
\endhead
\bottomrule\noalign{}
\endlastfoot
Control & 160.643 & 137.977 & 1161.76 & 385.253 & 482.63 & 301.24 \\
Treated & 605.979 & 381.52 & 1964.58 & 833.028 & 983.46 & 506.70 \\
\end{longtable}

--- end-multi-column

\%\% Small sample size, Group overlap, predictive power \%\% \%\% Look
at these overlapping reigions \%\% \%\% Cohen's d = 1.2015293213102536,
this shows a strong effect size, but this \%\% \#\#\#\#\#\# Figure:
Force Indentation for Apparent Cell YM Coloured by Test Group

{} \%\% {} \%\%

Figure: Young's Module by Group with Confidence Metrics

{}

In the figure above the control group is contrasted with the treated
group using notched Tukey style box plots overlaid with the mean and
standard deviation 95\% confidence intervals as well as the apparent
cell YM values for each group.

\%\% Overlapping notches, what dose this mean for the usability of the
metric \%\%

\%\% {} \%\%

\subsection{Discussion}\label{discussion}

\%\% Best and worst cases based on confidence intervals \%\%

The sample sizes used in this report are not sufficient to produce a
classifier suitable for use in industry or for diagnosis, considering
the the confidence intervals of the groups established it is possible
that a larger experimental dataset my prove this method to be
significantly more or less effective than has been estimated here. By
considering the limit cases of the 95\% confidence intervals of the most
distinct best case i.e.~furthest means and smallest standard deviations
this method may prove highly accurate requiring very few samples in
contrast the worst case least distinct i.e.~closest means and wisest
spread would render this method completely ineffective.

Figure: Best vs Worst Case Probability Density Functions

{}

It should be noted that with the small sample size of the treated group
n=4, these metrics are significantly brought upward by the results of a
single cell. However, other than the substantially higher Young's Module
values there is no reason to expect this cell or it's experiments to be
erroneous. The Hetz fit's appear representative of the observed cell
response with residuals similar to average across all fits 2e-11 N.

In both the Control and the Treated group the majority of cells
consistently exhibit YM lower than the group average with a few very
high YM cells with higher inter experiment range. However, the range in
YM increases linearly with with higher average YM values, when variance
is considered proportionally there is no correlation, so this can likely
be considered systematic error.

Due to the unexplained variance in the range of YM across tests of
single cells the possibility of it's relation to the diseased state has
not been ruled out Introducing the possibility that method increases
classification accuracy at the cell level but potentially sacrifices it
at the population level. This provides an argument to establish group
characteristics on the experiment level rather than the cell level for
use cases where many samples are being taken from single unknown group
as might be the case in a biopsy for example.

\subsection{Conclusion}\label{conclusion}

\subsection{Annexes}\label{annexes}

\subsubsection{Control}\label{control}

{} {} {} {} {} {} {} {} {} {} {}

\subsubsection{Treated}\label{treated}

{} {} {} {}

\end{document}